\documentclass[article,oneside, 12pt]{abntex2}

\usepackage[alf]{abntex2cite}
\usepackage[T1]{fontenc}
\usepackage[utf8]{inputenc}
\usepackage{tikz}
\usepackage{lipsum}
\usepackage[compact]{titlesec}
\usepackage{booktabs}
\usepackage{multirow}
\usepackage{graphicx}
\usepackage[normalem]{ulem}
\useunder{\uline}{\ul}{}


\usepackage{times}
\renewcommand{\ABNTEXsectionfont}{\fontfamily{cmr}\fontseries{b}\selectfont}
\renewcommand{\ABNTEXsectionfontsize}{\Large}
\titlespacing{\section}{0pt}{5pt}{10pt}
\setlrmarginsandblock{3cm}{2cm}{*}
\setulmarginsandblock{3cm}{2cm}{*}
\checkandfixthelayout
\setlength{\parskip}{12pt}
\renewcommand{\baselinestretch}{1.5}

\usepackage{fancyhdr}

\pagestyle{fancyplain}
\fancyhead{}
\fancyfoot{}
\fancyfoot[R]{\thepage}


%!TEX root = projeto.tex

\titulo{Título}
\autor{Autor}
\local{Cidade}
\data{Ano}
%\orientador{Prof. Dr. Ariel Levy}
%Variáveis determinantes para formação do conceito preliminar de curso nas avaliações do Enade
\instituicao{Instituição}
\tipotrabalho{Tipo de trabalho}
\preambulo{Lorem ipsum dolor sit amet, consectetur adipiscing elit, sed do eiusmod tempor incididunt ut labore et dolore magna aliqua. Ut enim ad minim veniam, quis nostrud exercitation ullamco laboris nisi ut aliquip ex ea commodo consequat.}
\makeatletter
  \hypersetup{
  pdftitle={\@title},
  pdfauthor={\@author},
  pdfsubject={\imprimirpreambulo},
  pdfkeywords={PALAVRAS}{CHAVE}{EM}{PORTUGUES},
  pdfcreator={LaTeX with abnTeX2},
  colorlinks=true,
  linkcolor=blue,
  citecolor=blue,
  urlcolor=blue
  }
\makeatother

\begin{document}

%capa modelo uff
\begin{tikzpicture}[remember picture, 
  overlay, 
  rotate=-3.3, 
  scale=1.3,
  transform shape]
    \node[inner sep=0pt] at (current page.center) {
      \hspace*{-1.4cm}
      \includegraphics[scale=1]{resources/uff-cover.pdf}
    };
\end{tikzpicture}
\imprimircapa

\input{inputs/cover-sheet}

%%%%%%%%%%%%%%%%%%%%%%%%%%%%%%%%%%
\textual

\section{Sit amet nibh feugiat semper.}

% lorem{{{
Tincidunt. Ut consequat nisi sit amet nibh. Nunc mi tortor, tristique
sit amet, rhoncus porta, malesuada elementum, nisi. Integer vitae enim
quis risus aliquet gravida. Curabitur vel lorem vel erat dapibus
lobortis. Donec dignissim tellus at arcu. Quisque molestie pulvinar
sem. \cite[p.~42]{latexcompanion}

Nulla magna neque, ullamcorper tempus, luctus eget, malesuada ut,
velit. Morbi felis. Praesent in purus at ipsum cursus posuere. Morbi
bibendum facilisis eros. Phasellus aliquam sapien in erat. Praesent
venenatis diam dignissim dui. Praesent risus erat, iaculis ac, dapibus
sed, imperdiet ac, erat. Nullam sed ipsum. Phasellus non dolor. Donec
ut elit.

Sed risus.

Lorem ipsum dolor sit amet, consectetuer adipiscing elit. Vestibulum.
% lorem}}}


\pagebreak
\section{Aliquam aliquam dolor at justo.}

% lorem{{{
Mollis posuere. Quisque ac quam sed \citeonline[p.~42]{latexcompanion} massa adipiscing rutrum.
Vestibulum ipsum. Phasellus porta sapien. Maecenas venenatis tellus
vel tellus.

Aliquam aliquam dolor at justo. Cum sociis natoque penatibus et magnis
dis parturient montes, nascetur ridiculus mus. Morbi pretium purus a
magna. Nullam dui tellus, blandit eu, facilisis non, pharetra
consectetuer, leo. Maecenas.
% lorem}}}


\pagebreak
\section{Duis velit magna, scelerisque vitae}

% lorem{{{
Egestas. Praesent lacus diam, auctor quis, venenatis in, hendrerit at,
est. Vivamus eget eros. Phasellus congue, sapien ac iaculis feugiat,
lacus lacus accumsan lorem, quis volutpat justo turpis ac mauris.

Duis velit magna, scelerisque vitae, varius ut, aliquam vel, justo.
Proin ac augue. Nullam auctor lectus vitae arcu. Vestibulum porta
justo placerat purus. Ut sem nunc, vestibulum nec, sodales vitae,
vehicula eget.
% lorem}}}



\clearpage
%%%%%%%%%%%%%%%%%%%%%%%%%%%%%%%%%%

\pagebreak

% Replace with yours
%\bibliography{../../Referencias/refs-SP1.bib}
\bibliography{references}

\end{document}
